%!TEX TS-program = LuaLaTeX
%!BIB TS-program = biber


\documentclass[11pt,a4paper,sans]{moderncv}

\moderncvstyle{classicbank}
\moderncvcolor{maroon}

\newif\ifselectedpub
\newif\iffull


\selectedpubtrue 
%\fulltrue




\nopagenumbers{}

\usepackage[scale=0.75]{geometry}

\usepackage{etoolbox}
\usepackage{tcolorbox}
\usepackage{fontawesome}
\usepackage{graphicx}
\graphicspath{{./images/}}



\newcommand*{\collapseline}{\vspace{-1\normalbaselineskip}}

\newtcbox{\tech}{on line, colframe=color1, boxrule=0.4pt, arc=0mm, left=2pt, right=2pt, top=0pt, bottom=0pt, boxsep=0pt, before upper=\strut}
\newtcbox{\toolbox}{nobeforeafter, tcbox raise=-8pt , colframe=black, boxrule=0.1pt, arc=0mm, left=2pt, right=2pt, top=0pt, bottom=0pt, boxsep=0pt, before upper=\strut}
\newcommand{\tool}[2]{\href{#1}{\toolbox{#2}}}

\newcommand{\tpharo}{\href{http://pharo.org}{\tech{Pharo}}~}
\newcommand{\tmoose}{\href{http://moosetechnology.org}{\tech{Moose}}~}
\newcommand{\troassal}{\href{http://agilevisualization.com}{\tech{Roassal}}~}

\newcommand{\sobadge}[1]{
\raisebox{-.3\height}{\includegraphics[height=2\fontcharht\font`\B]{#1}}}


\newcommand*{\serviceentry}[3]{
	\cventry{#1\collapseline}{}{#2}{}{}{#3}\vspace{0.1\normalbaselineskip}
  }

\newcommand*{\teachentry}[4]{
	\cventry{#1\collapseline}{}{#3 \textnormal{as {\itshape #2}}}{}{}{#4}\vspace{0.1\normalbaselineskip}
  }

\newcommand*{\jobentry}[5]{
	\cventry{#1\collapseline}{}{#3 \textnormal{as {\itshape #2}}}{}{}{#4 \notblank{#5}{\\\texttt{\footnotesize #5}}{}}\vspace{0.1\normalbaselineskip}
  }

\newcommand*{\eduentry}[4]{
	\cventry{#1\collapseline}{}{#3 \textnormal{@ {\itshape #2}}}{}{}{#4}\vspace{0.1\normalbaselineskip}
  }


\name{Dr. Yuriy}{Tymchuk}
\title{Big Data Engineer}                               % optional, remove / comment the line if not wanted
\address{Waldeggstrasse 51}{CH-3097 Liebefeld}{Switzerland}% optional, remove / comment the line if not wanted; the "postcode city" and and "country" arguments can be omitted or provided empty
%\phone[fixed]{+41~(0)58~666~4293}                    % optional, remove / comment the line if not wanted
\email{yuriy.tymchuk@swisscom.com}                               % optional, remove / comment the line if not wanted
\homepage{yuriy.tymch.uk}                         % optional, remove / comment the line if not wanted
\extrainfo{}                 % optional, remove / comment the line if not wanted
\photo[80pt][0pt]{picture_round}                       % optional, remove / comment the line if not wanted; '64pt' is the height the picture must be resized to, 0.4pt is the thickness of the frame around it (put it to 0pt for no frame) and 'picture' is the name of the picture file
%\quote{Some quote}                                 % optional, remove / comment the line if not wanted

\usepackage{textcomp}
\usepackage{csquotes}
\usepackage[backend=biber,
	maxnames=99,
	bibstyle=authortitle,
	isbn=false,
	doi=false,
	url=false,
	sorting=ydnt,
	defernumbers=true]{biblatex}
\addbibresource{publications.bib}

\renewcommand*{\newunitpunct}{\addcomma\space}

\DeclareFieldFormat
  [inproceedings, thesis]
  {title}{\textbf{#1\isdot}}

\DeclareFieldFormat
  {booktitle}{In #1}

\DeclareNameFormat{author}{%
		\textit{%
        \nameparts{#1}%
        \ifuseprefix
          {\usebibmacro{name:given-family}
          {\namepartfamily}
          {\namepartgiveni}
          {\namepartprefix}
          {\namepartsuffixi}}
        {\usebibmacro{name:given-family}
          {\namepartfamily}
          {\namepartgiveni}
          {\namepartprefixi}
          {\namepartsuffixi}}%
      \usebibmacro{name:andothers}}}
   

\DeclareBibliographyDriver{inproceedings}{%
  \printfield{title}%
  \newunit\newblock
  \printnames{author}%
  \newunit\newblock
  \printfield{booktitle}%
  \newunit
  \printfield{pages}%
  \newunit
  \printlist{publisher}%
  \newunit
  \printfield{year}%
  \finentry}

\DeclareBibliographyDriver{thesis}{%
  \printfield{title}%
  \addspace
  \printtext{(Mater's thesis)}%
  \newunit\newblock
  \printnames{author}%
  \newunit\newblock
  \printlist{institution}%
  \newunit
  \printlist{location}%
  \newunit
  \printfield{year}%
  \finentry}

\defbibfilter{conferences}{type=inproceedings and subtype=conference}
\defbibfilter{demo}{type=inproceedings and subtype=demo}
\defbibfilter{doc-symposium}{type=inproceedings and subtype=doc-sym}
\defbibfilter{workshops}{type=inproceedings and subtype=workshop}
\defbibfilter{theses}{type=thesis}
\defbibfilter{selected}{keyword=selected}

\newcommand{\printbibsection}[2]{
  \begin{refsection}
    \nocite{*}
    \printbibliography[title={#1}, filter={#2}, heading=subbibliography]
  \end{refsection}
}

\quote{I'm a software engineer / computer scientist and a hacker in general, who loves to solve challenging problems. Programming is the main activity that helps me to bring my solutions alive.}


\makeatletter\chardef\pdf@shellescape=\@ne\makeatother


\begin{document}

\makecvtitle

\section{Achievements}

\cvitem{\href{https://github.com/Uko/QualityAssistant}{QualityAssistant}}{My software-quality tool with a live feedback is integrated into the latest version of \href{http://pharo.org}{Pharo IDE}. Expected usage: 30 commercial companies, 20 universities (for teaching).}
\cvitem{Live 3D visualizations video \textnormal{\href{https://youtu.be/CuimMwuZiGA}{[youtu.be/CuimMwuZiGA]}}}{was used in a couple of dozen presentations for demonstrating the idea of live programming.}
\cvitem{\href{https://stackoverflow.com/users/982238/uko}{StackOverflow}}{Owner of \href{https://stackoverflow.com/help/badges/1829/pharo}{\sobadge{pharo-badge}} {\footnotesize (9 owners in total)} and \href{https://stackoverflow.com/help/badges/2904/smalltalk}{\sobadge{smalltalk-badge}} {\footnotesize (3 owners in total)} badges.}


\section{Employment}
\jobentry{Dec 2017 -- ongoing}{Big Data Engineer {\small (+ Scrum Master)}}{\href{http://swisscom.ch}{Swisscom}}{Development of an ETL application, refactoring of a client-facing web app, maintenance of a CI/DC infrastructure. Reduced the deployment time from one human/day to a few minutes with minimum interactions. Saved a few dying features and operational failures.}{\tech{Scala} \tech{Spark} \tech{NodeJS} \tech{Solr} \tech{Jenkins} \tech{Cloud Foundry}}
\jobentry{Jan 2016 -- Nov 2017}{Research Assistant}{\href{http://www.unibe.ch}{Universit{\"a}t Bern}}{
Research on embedding quality assistance into development workflow. Maintenance of the group's git server. I've invented a 3D visualization that allows detection of defects in software quality evolution. Teaching.}{\tpharo \tmoose \troassal \tech{Python} \tech{Ruby} \tech{Java} \tech{git}}
\jobentry{Jun 2015 -- ongoing}{Lecturer \& Strategic Consultant}{\href{http://ucu.edu.ua/eng/}{Ukrainian Catholic University}}{
Planning the first year for the bachelor computer science program and a long term strategy for the Faculty of Applied Sciences. Teaching advanced programming for the data science master students.}{}
\jobentry{Oct 2013 -- Dec 2015}{Doctoral Assistant}{\href{http://www.usi.ch}{Universit\`a della Svizzera Italiana}}{
Modeling of developer collaboration, based on the data mined from software repository.
Research on code quality and code review, development of \href{http://vidi.inf.usi.ch}{ViDI} code review tool and \href{https://github.com/Uko/QualityAssistant}{QualityAssistant} static analyzer. Teaching.}{\tpharo \tmoose \troassal \tech{Gephi} \tech{Python} \tech{Racket} \tech{MongoDB}}
\jobentry{Apr 2013 -- Sep 2013}{Software engineer}{\href{http://www.innocode.no}{Innocode}}{
Pre-project research concerning a \href{http://www.leanbusiness.no/en/}{leanbusiness.no} web-application.
Development of \href{http://relink.no}{Relink} project in Ruby on Rails. Increased performance of the team by introducing git-flow strategy.}{\tech{Ruby on Rails} \tech{JavaScript} \tech{HTML/CSS} \tech{git}}
\jobentry{Jan 2013 -- Apr 2013}{Intern}{\href{http://rmod.lille.inria.fr/web/}{RMoD}, \href{https://www.inria.fr}{Inria}}{
Research on AST metamodels. Development of FAST: a generic metamodel and symbol resolution algorithm.}{\tpharo \tmoose \tech{PetitParser}}
\jobentry{Aug 2012 -- Jan 2013}{Software engineer}{\href{http://interlogic.com.ua}{InterLogic}}{
Development of a Java3D applet for ventilation system configuration with a backend in PHP and MySQL. I've developed a module in Scala which decreased the complexity by an order of magnitude in comparison with Java implementation.}{\tech{Java} \tech{Java3D} \tech{Scala} \tech{PHP} \tech{MySQL}}
\jobentry{Mar 2010 -- Aug 2012}{Founder/Leader}{\href{http://unikernel.net}{Unikernel Team} \textnormal{\emph{(freelance)}}}{Different websites made from scratch. Usually online shops, sometimes web apps with rich frontend.}{\tech{HTML/CSS} \tech{PHP} \tech{CakePHP} \tech{JavaScript} \tech{Raphaël JS} \tech{MySQL}}
\jobentry{Feb 2009 -- Jul 2012}{Network Administrator}{\href{http://www.uar.net/en/}{UARNet ISP}}{On call assistance, network administration, minor web development.}{\tech{FreeBSD} \tech{Linux} \tech{iptables} \tech{Oracle RDBMS} \tech{PHP}}

%\section{Programming Skills}
%\cvitem{Experienced}{Pharo/Smalltalk, Ruby+Rails, Java, git, TDD, Software Maintainability}
%\cvitem{Familiar}{HTML, CSS, JavaScript, Scala, Objective-C, Python, C++, Lisp, MySQL, MongoDB, Mercurial, SVN}


\section{Education}

\eduentry{2013 -- 2017}{University of Lugano \& University of Bern}{Ph.D. in Computer Science}{Thesis topic: ``Quality-Aware Tooling''}

\eduentry{2012 -- 2013}{Ivan Franko National University of Lviv}{Master of Science}{Specialization in Informatics}
\eduentry{2008 -- 2012}{Ivan Franko National University of Lviv}{Bachelor of Science}{Specialization in Informatics}
\iffull
\eduentry{2006 -- 2008}{Lviv Physics and Mathematics Lyceum}{Highschool diploma}{Specialization in Physics and Mathematics}
\eduentry{2006 -- 2008}{Minor Academy of Sciences of Ukraine}{Member of MASU}{``Whimsical Bride'' - the graduation research project about statistics theory}
\fi

\section{Additional Training}
\eduentry{2018}{Swisscom}{\href{https://www.youracclaim.com/badges/368292d3-f771-4bee-9c76-74a4d8e4a732}{SAFe Scrum Master}}{}
\eduentry{2017}{CUSO Winter School}{\href{https://www.cuso.ch/activity/?p=2283\&uid=3307}{Design Science \& Design Thinking}}{}
\eduentry{2016}{CUSO Seminar}{\href{https://www.cuso.ch/activity/?p=2283\&uid=3018}{Hands-on Data Analysis with R}}{}
\eduentry{2016}{CUSO Winter School}{\href{https://www.cuso.ch/activity/?p=2283\&uid=3022}{Taming Big Data}}{}
\eduentry{2014}{International Summer School on Software Eng.}{\href{http://www.sesa.unisa.it/seschool/previousEditions/2014/}{Search-Based Software Engineering}}{}


\section{Latest Tools, Frameworks, and Libraries}

\faArrowCircleORight \  Most of my project are available at my GitHub account \href{https://github.com/Uko}{[github.com/Uko]}

\tool{https://github.com/Uko/GitHubcello}{GitHubcello}
\tool{https://github.com/Uko/DeprecationAssistant}{DeprecationAssistant}
\tool{https://github.com/Uko/MatchTool}{MatchTool}
\tool{https://github.com/Uko/QualityAssistant}{QualityAssistant}
\tool{http://yuriy.tymch.uk/Renraku/}{Renraku}
\tool{http://yuriy.tymch.uk/Vidi}{ViDI}
\tool{http://yuriy.tymch.uk/Smalldromeda}{Smalldromeda}


\nocite{*}

\section{\ifselectedpub Selected \fi Publications}

\faArrowCircleORight \  Besides academic publishing I have a coding blog: \href{http://code.yuriy.tymch.uk}{[code.yuriy.tymch.uk]}


\ifselectedpub
	\vspace{-8pt}
	\newrefcontext[sorting=ydnt]
	\printbibliography[keyword=selected, heading=none]
\else
	\printbibsection{conferences}{conferences}
	\printbibsection{formal demonstrations}{demo}
	\printbibsection{doctoral symposium}{doc-symposium}
	\printbibsection{workshops}{workshops}
	\printbibsection{theses}{theses}  
\fi

                  


\section{Teaching}

\teachentry{}{Lecturer}{Ukrainian Catholic University}{
Advanced Programming\hfill {\itshape Fall 2016}}
\teachentry{}{Teaching Assistant}{Universit{\"a}t Bern}{
Compiler Construction\hfill {\itshape Spring 2017}\\
Software Modeling and Analysis\hfill {\itshape Fall 2016}}
\teachentry{}{Teaching Assistant}{Universit\`a della Svizzera Italiana}{
Bachelors Project\hfill {\itshape Spring 2015}\\
Programming Fundamentals I\hfill {\itshape Fall 2014, Fall 2015}\\
Software Atelier IV\hfill {\itshape Spring 2014}}


\section{Service}
\serviceentry{}{[IWST] International Workshop on Smalltalk Technologies}{
Program Committee \hfill {\itshape 2017}}
\serviceentry{}{[VISSOFT] IEEE Working Conference on Software Visualization}{
Program Committee - NIER and Tool Demo Track \hfill {\itshape 2018}\\
Program Committee - NIER and Tool Demo Track \hfill {\itshape 2017}\\
Program Committee - NIER and Tool Demo Track \hfill {\itshape 2016}\\
Program Committee - Artifact Track      \hfill {\itshape 2015}}
\serviceentry{}{[GSoC] Google Summer of Code}{
Pharo Consortium Organization Admin and Student's Mentor \hfill {\itshape 2017}\\
European Smalltalk Users Group Student's Mentor         \hfill {\itshape 2015}}



\section{Personal Information}
\cvdoubleitem{Date of Birth}{Aug 11, 1991}{Marital Status}{Married}
\cvdoubleitem{Place of Birth}{Lviv, Ukraine}{\textnumero~of children}{1}
\cvitem{Citizenship}{Ukrainian}

\section{Languages}
\cvdoubleitem{English}{Proficient}{Polish (spoken)}{Intermediate}
\cvdoubleitem{Ukrainian}{Native Speaker}{Russian (spoken)}{Intermediate}
\cvitem{Italian}{Basic}

\section{Hobby}
Hiking picturesque or difficult mountains and reflecting amazing moments of this world with my camera (\url{https://500px.com/yuriy_tymchuk}). I own a 3D printer, and I design functional models that I print afterwords. Some time ago I was doing Kyokushinkai karate and I'm still following some of the principals I've learned there.

\section{Remarks for Nerds}
This cv is written in \LaTeX, versioned with git, hosted on \href{https://github.com/Uko/cv}{GitHub} and built with \href{https://travis-ci.org/Uko/cv}{Travis CI}


\end{document}


%% end of file `template.tex'.
