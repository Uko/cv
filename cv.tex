%!BIB TS-program = biber


\documentclass[11pt,a4paper,sans]{moderncv} 

\moderncvstyle{classicbank}
\moderncvcolor{maroon}

\nopagenumbers{}      
\usepackage[utf8]{inputenc}

\usepackage[scale=0.75]{geometry}

\name{Yuriy}{Tymchuk}
\title{Doctoral Assistant}                               % optional, remove / comment the line if not wanted
\address{Universit\`a della Svizzera Italiana}{Via G. Buffi 13}{6904 Lugano, Switzerland}% optional, remove / comment the line if not wanted; the "postcode city" and and "country" arguments can be omitted or provided empty
\phone[fixed]{+41~(0)58~666~4293}                    % optional, remove / comment the line if not wanted
\email{yuriy.tymchuk@usi.ch}                               % optional, remove / comment the line if not wanted
\homepage{yuriy.tymchuk.me}                         % optional, remove / comment the line if not wanted
%\extrainfo{additional information}                 % optional, remove / comment the line if not wanted
\photo[74pt][0pt]{picture_round}                       % optional, remove / comment the line if not wanted; '64pt' is the height the picture must be resized to, 0.4pt is the thickness of the frame around it (put it to 0pt for no frame) and 'picture' is the name of the picture file
%\quote{Some quote}                                 % optional, remove / comment the line if not wanted

\usepackage{csquotes}
\usepackage[backend=biber,
	maxnames=99,
	bibstyle=authortitle,
	isbn=false,
	doi=false,
	url=false,
	sorting=ydnt]{biblatex}
\addbibresource{publications.bib}

\renewcommand*{\newunitpunct}{\addcomma\space} 

\DeclareFieldFormat
  [inproceedings, thesis]
  {title}{\textbf{#1\isdot}}

\DeclareFieldFormat
  {booktitle}{In #1}

\DeclareNameFormat{author}{%
	\textit{
	\ifuseprefix
		{\usebibmacro{name:first-last}{#1}{#4}{#5}{#8}}
		{\usebibmacro{name:first-last}{#1}{#4}{#6}{#8}}%
   \usebibmacro{name:andothers}}}

\DeclareBibliographyDriver{inproceedings}{%
  \printfield{title}%
  \newunit\newblock
  \printnames{author}%
  \newunit\newblock
  \printfield{booktitle}%
  \newunit
  \printfield{pages}%
  \newunit
  \printlist{publisher}%
  \newunit
  \printfield{year}%
  \finentry}
  
\DeclareBibliographyDriver{thesis}{%
  \printfield{title}%
  \newunit\newblock
  \printnames{author}%
  \newunit\newblock
  \printlist{institution}%
  \newunit
  \printlist{location}%
  \newunit
  \printfield{year}%
  \finentry}

\defbibfilter{conferences}{type=inproceedings and subtype=conference}
\defbibfilter{demo}{type=inproceedings and subtype=demo}
\defbibfilter{workshops}{type=inproceedings and subtype=workshop}
\defbibfilter{theses}{type=thesis}

\newcommand{\printbibsection}[2]{
  \begin{refsection}
    \nocite{*}
    \printbibliography[title={#1}, filter={#2}, heading=subbibliography]
  \end{refsection}
}
\quote{I'm a software engineer / computer scientist who just loves innovation, software development methodologies, teamwork and open source.\\
I cannot affiliate myself with any programing language in particular as I believe that writing a program is an art of shaping a software system, and programming languages are just the tools.
However at the moment I believe that the best tool is Pharo/Smalltalk.\\
I like to lead, I like to make experiments, I hate when people are treated as resources.}





\begin{document}

\makecvtitle

\section{Experience}
\cventry{Oct 2013--Present}{Doctoral Assistant}{Universit\`a della Svizzera Italiana}{}{}{
Modeling of software developers collaboration, based on the data mined from software repository.\\
Research on code quality assessment and code review, development of \href{http://vidi.inf.usi.ch}{ViDI} project. }
\cventry{Apr 2013--Sep 2013}{Software engineer}{\href{http://www.cmdigital.no}{City Media Digital AS}}{}{}{
Pre-project research concerning web-application for \href{http://www.leanbusiness.no/en/}{leanbusiness.no}.\\
Development of \href{http://relink.no}{Relink} project. Most of work was related to ruby on rails project, although javascript, html and so on parts were touched too. Contributed to git/CI workflow of the team.}
\cventry{Jan 2013--Apr 2013}{Intern}{\href{http://rmod.lille.inria.fr/web/}{RMoD}, \href{https://www.inria.fr}{Inria}}{}{}{
Research on AST metamodels. Development of FAST generic metamodel and symbol resolution algorithm.}
\cventry{Aug 2012--Jan 2013}{Software engineer}{\href{http://interlogic.com.ua}{InterLogic}}{}{}{
Development of a Java3D applet for ventilation system configuration, with a backend in PHP and MySQL.}
\cventry{Mar 2010--Aug 2012}{Founder/Leader}{Unikernel Team \textnormal{\emph{(freelance)}}}{}{}{Different websites made from scratch. Usually online shops, sometimes web apps with rich frontend.}
\cventry{Feb 2009--Jul 2012}{System Administrator}{\href{http://www.uar.net/en/}{Ukrainian Academic Research Network}}{}{}{Administration of the network, minor web development.}

\section{Skils}
\cvitem{Experienced}{Pharo/Smalltalk, Ruby+Rails, Java, Agile, git, TDD}
\cvitem{Familiar}{HTML, CSS, JavaScript, Scala, Objective-C, Python, C++, Lisp, MySQL, MongoDB, Mercurial, SVN}

\section{Education}
\cventry{2012--2013}{Ivan Franko National University of Lviv}{Master of Science}{}{}{Specialization in Informatics}
\cventry{2008--2012}{Ivan Franko National University of Lviv}{Bachelor of Science}{}{}{Specialization in Informatics}
\cventry{2006--2008}{Lviv Physics and Mathematics Lyceum}{Highschool diploma}{}{}{Specialization in Physics and Mathematics}
\cventry{2006--2008}{Minor Academy of Sciences of Ukraine}{Member of MASU}{}{}{``Whimsical Bride'' - the graduation research project about statistics theory}


\section{Selected Publications}
\nocite{Tymc2015a, Tymc2014a, benevol13}
\printbibliography[heading=none]                       % 'publications' is the name of a BibTeX file


\section{Personal Information}
\cvdoubleitem{Date of Birth}{Aug 11, 1991}{Citizenship}{Ukrainian}
\cvdoubleitem{Place of Birth}{Lviv, Ukraine}{Marital Status}{Maried}

\section{Languages}
\cvdoubleitem{English}{Proficient}{Polish (spoken)}{Intermediate}
\cvdoubleitem{Ukrainian}{Mother Tongue}{Russian (spoken)}{Intermediate}
\cvitem{Italian}{Basic}

\section{Hobby}
Mountaineering, skiing, photography


\end{document}


%% end of file `template.tex'.
