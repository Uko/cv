%!TEX TS-program = LuaLaTeX
%!BIB TS-program = biber


\documentclass[11pt,a4paper,sans]{moderncv}

\moderncvstyle{classicbank}
\moderncvcolor{maroon}

\newif\ifselectedpub
\newif\ifphdeducation


%\selectedpubtrue 
%\phdeducationtrue



\nopagenumbers{}

\usepackage[scale=0.75]{geometry}

\newcommand*{\collapseline}{\vspace{-1\normalbaselineskip}}

\newcommand*{\jobentry}[4]{
	\cventry{#1\collapseline}{}{#3 \textnormal{as {\itshape #2}}}{}{}{#4}\vspace{0.1\normalbaselineskip}
  }

\newcommand*{\eduentry}[4]{
	\cventry{#1\collapseline}{}{#3 \textnormal{@ {\itshape #2}}}{}{}{#4}\vspace{0.1\normalbaselineskip}
  }


\name{Yuriy}{Tymchuk}
\title{Research Assistant}                               % optional, remove / comment the line if not wanted
\address{Schützenmattstrasse 14}{CH-3012 Bern}{Switzerland}% optional, remove / comment the line if not wanted; the "postcode city" and and "country" arguments can be omitted or provided empty
%\phone[fixed]{+41~(0)58~666~4293}                    % optional, remove / comment the line if not wanted
\email{tymchuk@inf.unibe.ch}                               % optional, remove / comment the line if not wanted
\homepage{yuriy.tymch.uk}                         % optional, remove / comment the line if not wanted
\extrainfo{}                 % optional, remove / comment the line if not wanted
\photo[80pt][0pt]{picture_round}                       % optional, remove / comment the line if not wanted; '64pt' is the height the picture must be resized to, 0.4pt is the thickness of the frame around it (put it to 0pt for no frame) and 'picture' is the name of the picture file
%\quote{Some quote}                                 % optional, remove / comment the line if not wanted

\usepackage{textcomp}
\usepackage{csquotes}
\usepackage[backend=biber,
	maxnames=99,
	bibstyle=authortitle,
	isbn=false,
	doi=false,
	url=false,
	sorting=ydnt,
	defernumbers=true]{biblatex}
\addbibresource{publications.bib}

\renewcommand*{\newunitpunct}{\addcomma\space}

\DeclareFieldFormat
  [inproceedings, thesis]
  {title}{\textbf{#1\isdot}}

\DeclareFieldFormat
  {booktitle}{In #1}

\DeclareNameFormat{author}{%
		\textit{%
        \nameparts{#1}%
        \ifuseprefix
          {\usebibmacro{name:given-family}
          {\namepartfamily}
          {\namepartgiveni}
          {\namepartprefix}
          {\namepartsuffixi}}
        {\usebibmacro{name:given-family}
          {\namepartfamily}
          {\namepartgiveni}
          {\namepartprefixi}
          {\namepartsuffixi}}%
      \usebibmacro{name:andothers}}}
   

\DeclareBibliographyDriver{inproceedings}{%
  \printfield{title}%
  \newunit\newblock
  \printnames{author}%
  \newunit\newblock
  \printfield{booktitle}%
  \newunit
  \printfield{pages}%
  \newunit
  \printlist{publisher}%
  \newunit
  \printfield{year}%
  \finentry}

\DeclareBibliographyDriver{thesis}{%
  \printfield{title}%
  \addspace
  \printtext{(Mater's thesis)}%
  \newunit\newblock
  \printnames{author}%
  \newunit\newblock
  \printlist{institution}%
  \newunit
  \printlist{location}%
  \newunit
  \printfield{year}%
  \finentry}

\defbibfilter{conferences}{type=inproceedings and subtype=conference}
\defbibfilter{demo}{type=inproceedings and subtype=demo}
\defbibfilter{doc-symposium}{type=inproceedings and subtype=doc-sym}
\defbibfilter{workshops}{type=inproceedings and subtype=workshop}
\defbibfilter{theses}{type=thesis}
\defbibfilter{selected}{keyword=selected}

\newcommand{\printbibsection}[2]{
  \begin{refsection}
    \nocite{*}
    \printbibliography[title={#1}, filter={#2}, heading=subbibliography]
  \end{refsection}
}

\quote{I'm a software engineer / computer scientist and a hacker in general who likes researching new ideas, developing solutions, and teaching.\\
I am focusing on software maintenance and evolution issues. I invest time in development of the tools that assist software engineers with their work.}





\begin{document}

\makecvtitle

\section{Achievements}

\cvitem{\href{https://github.com/Uko/QualityAssistant}{QualityAssistant}}{My software-quality tool with a live feedback is integrated into the latest version of \href{http://pharo.org}{Pharo IDE}. Expected usage: 30 commercial companies, 20 universities (for teaching).}
%\cvitem{StackOverflow}{reputation}

\section{Employment}
\jobentry{Jan 2016 -- Present}{Research Assistant}{\href{http://www.unibe.ch}{Universit{\"a}t Bern}}{
Research on software quality and evolutionary monitoring. Maintenance of the group's git server. Assistance in teaching. I've invented a 3D visualization that allows detection of defects in software quality evolution.}
\jobentry{Oct 2013 -- Dec 2015}{Doctoral Assistant}{\href{http://www.usi.ch}{Universit\`a della Svizzera Italiana}}{
Modeling of developer collaboration, based on the data mined from software repository.
Research on code quality and code review, development of \href{http://vidi.inf.usi.ch}{ViDI} code review tool and \href{https://github.com/Uko/QualityAssistant}{QualityAssistant} static analyzer. Assistance in teaching. }
\jobentry{Apr 2013 -- Sep 2013}{Software engineer}{\href{http://www.innocode.no}{Innocode}}{
Pre-project research concerning web-application for \href{http://www.leanbusiness.no/en/}{leanbusiness.no}.
Development of \href{http://relink.no}{Relink} project in Ruby on Rails. Increased performance of the team by introducing git-flow strategy.}
\jobentry{Jan 2013 -- Apr 2013}{Intern}{\href{http://rmod.lille.inria.fr/web/}{RMoD}, \href{https://www.inria.fr}{Inria}}{
Research on AST metamodels. Development of FAST: a generic metamodel and symbol resolution algorithm.}
\jobentry{Aug 2012 -- Jan 2013}{Software engineer}{\href{http://interlogic.com.ua}{InterLogic}}{
Development of a Java3D applet for ventilation system configuration with a backend in PHP and MySQL. I've developed a module in Scala which decreased the complexity by an order of magnitude in comparison with Java implementation.}
\jobentry{Mar 2010 -- Aug 2012}{Founder/Leader}{\href{http://unikernel.net}{Unikernel Team} \textnormal{\emph{(freelance)}}}{Different websites made from scratch. Usually online shops, sometimes web apps with rich frontend.}
\jobentry{Feb 2009 -- Jul 2012}{Network Administrator}{\href{http://www.uar.net/en/}{UARNet ISP}}{On call assistance, network administration, minor web development.}

\section{Programming Skills}
\cvitem{Experienced}{Pharo/Smalltalk, Ruby+Rails, Java, git, TDD, Software Maintainability}
\cvitem{Familiar}{HTML, CSS, JavaScript, Scala, Objective-C, Python, C++, Lisp, MySQL, MongoDB, Mercurial, SVN}


\section{Education}
\ifphdeducation
\eduentry{2013 -- 2017}{University of Lugano \& University of Bern}{Ph.D. in Informatics \textnormal{\emph{(ongoing)}}}{Thesis topic: ``Treating Code Quality as a First Class Entity''}
\fi
\eduentry{2012 -- 2013}{Ivan Franko National University of Lviv}{Master of Science}{Specialization in Informatics}
\eduentry{2008 -- 2012}{Ivan Franko National University of Lviv}{Bachelor of Science}{Specialization in Informatics}
\eduentry{2006 -- 2008}{Lviv Physics and Mathematics Lyceum}{Highschool diploma}{Specialization in Physics and Mathematics}
\eduentry{2006 -- 2008}{Minor Academy of Sciences of Ukraine}{Member of MASU}{``Whimsical Bride'' - the graduation research project about statistics theory}


\nocite{*}

\ifselectedpub
	\section{Selected Publications}
	\printbibliography[sorting=chronological, keyword=selected, heading=none]
\else
	\section{Publications}
	\printbibsection{conferences}{conferences}
	\printbibsection{formal demonstrations}{demo}
	\printbibsection{doctoral symposium}{doc-symposium}
	\printbibsection{workshops}{workshops}
	\printbibsection{theses}{theses}  
\fi

                  


\section{Teaching}

\jobentry{}{Teacher}{Ukrainian Catholic University}{
Advanced Programming\hfill {\itshape Fall 2016}}

\jobentry{}{Teaching Assistant}{Universit{\"a}t Bern}{
Compiler Construction\hfill {\itshape Spring 2017}\\
Software Modeling and Analysis\hfill {\itshape Fall 2016}}

\jobentry{}{Teaching Assistant}{Universit\`a della Svizzera Italiana}{
Bachelors Project\hfill {\itshape Spring 2015}\\
Programming Fundamentals I\hfill {\itshape Fall 2014, Fall 2015}\\
Software Atelier IV\hfill {\itshape Spring 2014}}


\section{Service}
\cventry{VISSOFT 2016}{4rd IEEE Working Conference on Software Visualization}{Program Committee - NIER and Tool Tracks}{}{}{}
\cventry{VISSOFT 2015}{3rd IEEE Working Conference on Software Visualization}{Artifact Evaluation Committee}{}{}{}

\section{Personal Information}
\cvdoubleitem{Date of Birth}{Aug 11, 1991}{Marital Status}{Married}
\cvdoubleitem{Place of Birth}{Lviv, Ukraine}{\textnumero~of children}{1}
\cvitem{Citizenship}{Ukrainian}

\section{Languages}
\cvdoubleitem{English}{Proficient}{Polish (spoken)}{Intermediate}
\cvdoubleitem{Ukrainian}{Native Speaker}{Russian (spoken)}{Intermediate}
\cvitem{Italian}{Basic}

\section{Hobby}
Hiking picturesque or difficult mountains and reflecting amazing moments of this world with my camera. Few years ago I was still doing Kyokushinkai karate which allowed me to train my body as hard as I was training my mind.

\section{Remarks for Nerds}
This cv is written in \LaTeX, versioned with git, hosted on \href{https://github.com/Uko/cv}{GitHub} and built with \href{https://travis-ci.org/Uko/cv}{Travis CI}


\end{document}


%% end of file `template.tex'.
